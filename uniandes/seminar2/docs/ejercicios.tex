\documentclass[a4paper,11pt]{article}
\usepackage[utf8]{inputenc}
\usepackage{amsmath,amssymb,amsthm,enumerate,amsthm,flafter}
%\usepackage[spanish]{babel}


\newcommand{\R}{{\ensuremath{\mathbb{R}}}}
\newcommand{\N}{{\ensuremath{\mathbb{N}}}}
\newcommand{\Z}{{\ensuremath{\mathbb{Z}}}}
\renewcommand{\v}{{\ensuremath{\boldsymbol{v}}}}

\DeclareMathOperator{\sen}{sen}
\DeclareMathOperator{\senh}{senh}
\DeclareMathOperator{\arcos}{arccos}
\DeclareMathOperator{\nulidad}{nulidad}

\usepackage{tikz}
\usetikzlibrary{calc,through,backgrounds,positioning,shapes.misc}
\usetikzlibrary{arrows,decorations.pathmorphing,fit,petri}
 


\voffset=-0.1cm
\hoffset=-1.6cm
\textwidth 16.5cm
\textheight 22cm
  
%opening
\title{Seminario Geometría discreta: Teoría de Ehrhart}
\author{Ejercicios propuestos}
\date{}

\begin{document}
\maketitle
\section{Capítulo 1}
\begin{enumerate}
 \item Entender el Teorema 1.5 (Popovicius) que dice que el número de formas de sumar $n$ con monedas de $a$ y $b$ es 
 $$p_{a,b}(n)=\frac{n}{ab}-\bigg\{\frac{b^{-1}n}{a}\bigg\}-\bigg\{\frac{a^{-1}n}{b}\bigg\}+1$$
 donde $\{x\}=x-\lfloor x \rfloor$,  $bb^{-1}  \equiv 1 \pmod{a}$ y $aa^{-1}  \equiv 1 \pmod{b}.$
 
 Para esto hay dos posibilidades. La primera es tratar de verlo directamente, el caso de dos monedas no es tan complicado.
 La segunda posibilidad es seguir los pasos del libro y los ejercicios propuestos correspondientes (al final de la Sección 1.3, y los ejercicios 1.3 y 1.22). El método del libro usando fracciones parciales
 permite encontrar otras fórmulas par los casos de mas de dos monedas, como por ejemplo las fórmulas de $p_{\{a,b,c\}}(n)$ y $p_{\{a,b,c\}}(n)$ que se encuentran al final del Capítulo 1 (pag. 15). 
 
 
 \item Entender las demostraciones de los teoremas 1.2 y 1.3 usando el Teorema 1.5 (esto se puede encontrar en la Sección 1.4).  
\end{enumerate}

 
 \section{Capítulo 2}
\begin{enumerate}
\item Si definimos los números $A(d,k)$ mediante la relación $$\sum_{j\ge 0}j^dz^j=  \sum_{i=0}^d \frac{A(d, k) z^k}{(1 - z)^{d+1}},$$ demostrar que:
  \begin{enumerate}
   \item El polinomio $\sum_{i=0}^d {A(d, k) z^k}$ es el numerador de $$\bigg(z\frac{d}{dz}\bigg)^d\bigg(\frac{1}{1-z}\bigg)$$ (y por lo tanto tiene sentido la definición).
   \item Investigar otras propiedades de los números eulerianos $A(d,k)$.
\end{enumerate}

\item Revisar la demostración del Lemma 2.3 sobre los polinomios de Bernoulli.

\item Ver la Sección 2.5, en donde se calculan los polinomios de Ehrhart de los cross-politopos (octahedros generalizados).
\end{enumerate}






\end{document}

 
 
